\begin{frame}
  \frametitle{Solution for the plane wave part of the density}
  In this part we will take advantage of the transition from \emph{linalg}
  data layout to \emph{fft} data layout.
  
  \begin{block}{Basic Idea}
    We will temporarily modify the data to make it look like normal DFT data so
    that we can pass it to the following procedure as always.
    It consists in obtaining a diagonal matrix of occupation associated to
    a base of wave functions vectors.
  \end{block}
\end{frame}

\begin{frame}
  In practice we:
  \begin{itemize}
    \item Compute the eigen values and eigen vectors of the occupation matrix
    \item The eigen values will be used as diagonal occupations 
    \item Apply the matrix formed by the eigen vectors as a rotation matrix to
      the matrix formed by the Kohn-Sham vectors
    \item The result of the previous step is passed to the following code as the new
      Kohn-Sham base.
  \end{itemize}

  Then the rest of the computation behave just like it always has.
\end{frame}

\begin{frame}
  \frametitle{Solution for the PAW part of the density}
  \begin{itemize}
    \item Band distributed everywere \Rightarrow no tricks this time
  \end{itemize}
  But:
  \begin{itemize}
    \item PAW components are really few compared to planewaves
    \item PAW density computation is rather light
  \end{itemize}
  A carefully crafted set of point-to-point MPI communications will do the job
  just well.
\end{frame}

\SetKwFor{ForEach}{for each}{do}{}
\SetKwIF{If}{ElseIf}{Else}{if}{then}{elif}{else}{}

\begin{frame}[fragile]
  The algorithm is the following:
  \begin{algorithm}[H]
    \If{the current CPU uses correlated bands}{
      \ForEach{correlated band}{
        \If{the band is available}{
          Extract the data and put it in the buffer\;
          \ForEach{remote CPU}{
            \If{it uses correlated bands}{
              \If{it needs the current band}{
                send the band\;
              }
            }
          }
        } \Else {
          \If{this CPU need this band}{
            receive the band from the CPU that own it and put it in the buffer\;
          }
        }
      }
    }
  \end{algorithm}
  It prevents deadlocks and grant that data are available when they are used.
\end{frame}

\begin{frame}
  Some precision about the actual implementations:
  \begin{itemize}
    \item MPI communications are implemented as asynchrone communications they are
      initilized and then the computation can start with already available bands
    \item Since correlated bands form a block arround the Fermi level, not all CPUs
      are concerned
    \item This part could be optimized further but as we will see it does not worth it

  \end{itemize}
\end{frame}
