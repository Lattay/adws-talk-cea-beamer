\begin{frame}
  \frametitle{Solution for the plane wave part of the density}
  In this part we will take advantage of the transition from \emph{linalg}
  data layout to \emph{fft} data layout.
  
  \begin{block}{Basic Idea}
    We will temporarily modify the data to make it look like normal DFT data so
    that we can pass it to the following procedure as always.
    It consists in obtaining a diagonal matrix of occupation associated to
    a base of wave functions vectors.
  \end{block}
\end{frame}

\begin{frame}
  In practice we:
  \begin{itemize}
    \item Compute the eigen values and eigen vectors of the occupation matrix
    \item The eigen values will be used as diagonal occupations 
    \item Apply the matrix formed by the eigen vectors as a rotation matrix to
      the matrix formed by the Kohn-Sham vectors
    \item The result of the previous step is passed to the following code as the new
      Kohn-Sham base.
  \end{itemize}

  Then the rest of the computation behave just like it always has.
\end{frame}

\begin{frame}
  \frametitle{Solution for the PAW part of the density}
  In this case we cannot use the same kind of tricks because the bands are
  always distributed.\\
  However this part is quite light and does not deal with a lot of coefficients.\\
  A carefully crafted set of point-to-point MPI communications is therefor absolutly
  perfect for the job.
\end{frame}

\SetKwFor{ForEach}{for each}{do}{}
\SetKwIF{If}{ElseIf}{Else}{if}{then}{elif}{else}{}

\begin{frame}[fragile]
  The algorithm is the following:
  \begin{algorithm}[H]
    \If{the current CPU uses correlated bands}{
      \ForEach{correlated band}{
        \If{the band is available}{
          Extract the data and put it in the buffer\;
          \ForEach{remote CPU}{
            \If{it uses correlated bands}{
              \If{it needs the current band}{
                send the band\;
              }
            }
          }
        } \Else {
          \If{this CPU need this band}{
            receive the band from the CPU that own it and put it in the buffer\;
          }
        }
      }
    }
  \end{algorithm}
  It prevents deadlocks and grant that data are available when they are used.
\end{frame}
