\documentclass[22pt]{beamer}
\usepackage{algorithm2e}

\usetheme[numbering=progressbar]{focus}
\definecolor{main1}{RGB}{159, 0, 14}
\definecolor{main2}{RGB}{80, 0, 5}
\definecolor{background}{RGB}{242, 236, 236}

\newcommand{\nologo}{\logo{}}
\title{Extension of the computation of density to non-diagonal band occupations with the KGB parallelization}

\subtitle{.}

\author{T.~Cavignac}

\institute{
  CEA, DAM, DIF, 91297 Arpajon Cedex, France
  \and
  École Centrale de Lyon
}

\date{\textsc{Abinit} developer Workshop, Mai 2019}

\logo{\includegraphics[height=1.5cm]{pic/logo-cea.png}}

\begin{document}
  \frame{\titlepage}

  \nologo

  \begin{frame}
    \frametitle{Table of Contents}
    \tableofcontents
  \end{frame}

  \section{paral\_kgb and the repartition of data}
  \begin{frame}
  \frametitle{The DMFT, a non local density theory}
  \begin{itemize}
    \item Dynamic Mean Field Theory (DMFT): one of the theory developed
  to address the problem of correlated electrons in transition metals 
  and lanthanides
    \item deals with non-local density and is totally different from the DFT
  \end{itemize}
  How to integrate it with \textsc{Abinit} ?
  \begin{itemize}
    \item Define non-diagonal occupations of the Kohn-Sham vectors from the DFT
    \item Compute the density from these occupations and the Kohn-Sham vectors
  \end{itemize}
\end{frame}

\begin{frame}
  \frametitle{From DFT density to DMFT+DFT density}
  Density expressed in terms of Kohn-Sham vectors and occupations:
  \begin{equation}
    n(\vec{r}) = \sum_{i \in \text{bands}} d_i~\phi_i(\vec{r})^* \phi_i(\vec{r})
  \end{equation}

  With non-diagonals occupations

  \begin{equation}
    n(\vec{r}) = \sum_{i, i' \in \text{bands}} d_{i,i'}~\phi_{i'}(\vec{r})^* \phi_i(\vec{r})
  \end{equation}
  \alert{We have now to compute products of vectors from different bands}
\end{frame}


  \section{Non local density, DMFT and DFT}
  \begin{frame}
  \frametitle{Reminder on Abinit parallelization}
  
  The \texttt{paral\_kgb} mode imply:
  \begin{itemize}
    \item Parallelization over $k$ vectors, which is trivial because
    they are always orthogonal.
  \end{itemize}

  And depending on the context one of:

  \begin{itemize}
    \item Parallelization over \emph{plane waves} ($g$)
    \item Parallelization over \emph{bands} ($b$)
  \end{itemize}
  Also in some cases there is locally parallelization over atoms, PAW projectors...
\end{frame}

\begin{frame}
  \frametitle{Parallelization for the computation of the density (plane waves part)}
  \begin{itemize}
    \item At the begining of \texttt{mkrho} wave functions are represented in
      reciprocal space and parallelized over plane waves (we call this the \emph{linalg} layout).
    \item A transposition of coefficients is performed (changing the parallelization from planes waves to bands, the layout is called \emph{fft}) 
    \item A Fourier transform is applied (changing the representation from reciprocal space to real space)
    \item Density is efficiently computed in real space with a natural parallelization
      over bands 
  \end{itemize}
  
  \alert{}

% As the density is computed by making product of vectors in the real space,
% it would imply a convolution product in reciprocal space.
% As a consequence we do not have plane waves at the time we compute the density.
% Also the natural parallelization is over the band. In the classic DFT case it
% is straitforward since we simply split the sum between CPUs.\\
% In our case it is no longer simple since we need arbitrary pairs of bands on
% each CPUs.
\end{frame}

\begin{frame}
  \frametitle{Parallelization for the computation of the density (PAW part)}
  In PAW part there is by definitions no plane waves. Also PAW base has few vectors
  and parallelization over them is quite restricted.\\
  Then band parallelization is the default.
\end{frame}

\begin{frame}
  \begin{block}{Conclusion}
    Computation of the density from DFT+DMFT is not possible as is because a given
    CPU would need to access arbitrary pairs of bands at a time where bands are
    distributed over CPUs.
  \end{block}
\end{frame}


  \begin{frame}
    \begin{block}{Conclusion}
      Computation of the density from DFT+DMFT is not possible as is because a given
      CPU would need to access arbitrary pairs of bands at a time where bands are
      distributed over CPUs.
    \end{block}
  \end{frame}

  \section{Solving this incompatibility}
  \begin{frame}
  \frametitle{Solution for the plane wave part of the density}
  Take advantage of the initial state of data in \texttt{mkrho}
  
  \begin{block}{Goal}
    Temporarily modify the data to make it look like normal DFT data.
  \end{block}
\end{frame}

\begin{frame}
  In practice we:
  \begin{enumerate}
    \item Compute the eigen values and eigen vectors of the occupation matrix
    \item Eigen values replace classic occupations
    \item Apply the matrix formed by the eigen vectors as a rotation matrix to
      the matrix formed by the Kohn-Sham vectors
    \item The rotated Kohn-Sham vectors are used in place of the originals
  \end{enumerate}

  Then the rest of the computation behave just like it always has.
\end{frame}

\begin{frame}
  \frametitle{Solution for the PAW part of the density}
  \begin{itemize}
    \item Band distributed everywere \Rightarrow no tricks this time
  \end{itemize}
  But:
  \begin{itemize}
    \item PAW components are really few compared to planewaves
    \item PAW density computation is rather light
  \end{itemize}
  A carefully crafted set of point-to-point MPI communications will do the job
  just well.
\end{frame}

\SetKwFor{ForEach}{for each}{do}{}
\SetKwIF{If}{ElseIf}{Else}{if}{then}{elif}{else}{}

\begin{frame}[fragile]
  The algorithm is the following:
  \begin{algorithm}[H]
    \If{the current CPU uses correlated bands}{
      \ForEach{correlated band}{
        \If{the band is available}{
          Extract the data and put it in the buffer\;
          \ForEach{remote CPU}{
            \If{it uses correlated bands}{
              \If{it needs the current band}{
                send the band\;
              }
            }
          }
        } \Else {
          \If{this CPU need this band}{
            receive the band from the CPU that own it and put it in the buffer\;
          }
        }
      }
    }
  \end{algorithm}
  It prevents deadlocks and grant that data are available when they are used.
\end{frame}

\begin{frame}
  Some precision about the actual implementations:
  \begin{itemize}
    \item MPI communications are implemented as asynchrone communications they are
      initialized and then the computation can start with already available bands
    \item Since correlated bands form a block arround the Fermi level, not all CPUs
      are concerned
    \item This part could probably be optimized further but as we will see it does not worth it

  \end{itemize}
\end{frame}


  \section{Validation process and results}
  \begin{frame}
  \frametitle{Validation process}
  \begin{itemize}
    \item Comparison of the results with the new method and the old one at the
      11th decimal of total energy in a few test cases with up to 100 steps
    \item Comparison of various intermediates quantities on the first iterations
    \item Comparison of the results with differents diagonalization algorithm
      for the Hamiltonian (LOBPCG, Chebychev, Conjugate gradient)
    \item Comparison of the results with differents CPU configurations
    \item Add of \texttt{paral[84]} and \texttt{paral[86]} to the testsuite
  \end{itemize}
\end{frame}

\begin{frame}
  \frametitle{Results}
  \begin{figure}[ht]
    \centering
    \includegraphics[width=0.7\textwidth]{pic/bargraph.png}
    \caption{Drastic effect of the use of paral\_kgb on a DMFT computation}
    \label{fig:bargraph.png}
  \end{figure}
\end{frame}


  \begin{frame}
    Finally I want to express my gratitude to Marc Torrent, Jordan Bieder, for the time
    they gave me to answer my questions and facilitate my discovery of \textsc{Abinit}, its code
    and its ecosystem.\\
    Of course, I also want to sincerely thank Bernard Amadon for his great supervision all along this work.

  \end{frame}

  {\nologo
    \begin{frame}[focus]
      Thank you for your attention !
    \end{frame}

    \begin{frame}[focus]
      Questions ?
    \end{frame}
  }

\end{document}
