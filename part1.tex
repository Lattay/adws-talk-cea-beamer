\begin{frame}
  \frametitle{The DMFT, a non local density theory}
  The Dynamic Mean Field Theory (DMFT) is one of the theory developed
  to address the problem of correlated electrons in transition metals 
  and lanthanides.\\
  It deals with non-local density and is totally different from the DFT.\\
  To integrate it to \textsc{Abinit}, its output is represented as a non
  diagonal matrix of occupations of the Kohn-Sham vectors from the DFT part of \textsc{Abinit}.\\
  Then the density is computed from the Kohn-Sham vectors and this matrix.
\end{frame}

\begin{frame}
  \frametitle{From DFT density to DMFT+DFT density}
  Density expressed in terms of Kohn-Sham vectors and occupations:
  \begin{equation}
    n(\vec{r}) = \sum_{i \in \text{bands}} d_i~\phi_i(\vec{r})^* \phi_i(\vec{r})
  \end{equation}

  With non-diagonals occupations

  \begin{equation}
    n(\vec{r}) = \sum_{i, i' \in \text{bands}} d_{i,i'}~\phi_{i'}(\vec{r})^* \phi_i(\vec{r})
  \end{equation}
  \alert{We have now to compute products of vectors from different bands}
\end{frame}
