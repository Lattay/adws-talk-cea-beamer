\begin{frame}
  \frametitle{Reminder on Abinit parallelization}
  
  The \texttt{paral\_kgb} mode imply:
  \begin{itemize}
    \item Parallelization over $k$ vectors, which is trivial because
    they are always orthogonal.
  \end{itemize}

  And depending on the context one of:

  \begin{itemize}
    \item Parallelization over \emph{plane waves} ($g$)
    \item Parallelization over \emph{bands} ($b$)
  \end{itemize}
  Also in some cases there is locally parallelization over atoms, PAW projectors...
\end{frame}

\begin{frame}
  \frametitle{Parallelization for the computation of the density (plane waves part)}
  \begin{itemize}
    \item At the begining of \texttt{mkrho} wave functions are represented in
      reciprocal space and parallelized over plane waves (we call this the \emph{linalg} layout).
    \item A transposition of coefficients is performed (changing the parallelization from planes waves to bands, the layout is called \emph{fft}) 
    \item A Fourier transform is applied (changing the representation from reciprocal space to real space)
    \item Density is efficiently computed in real space with a natural parallelization
      over bands 
  \end{itemize}
  
  \alert{}

% As the density is computed by making product of vectors in the real space,
% it would imply a convolution product in reciprocal space.
% As a consequence we do not have plane waves at the time we compute the density.
% Also the natural parallelization is over the band. In the classic DFT case it
% is straitforward since we simply split the sum between CPUs.\\
% In our case it is no longer simple since we need arbitrary pairs of bands on
% each CPUs.
\end{frame}

\begin{frame}
  \frametitle{Parallelization for the computation of the density (PAW part)}
  In PAW part:
  \begin{itemize}
    \item no plane waves by definition
    \item few components in the PAW base
  \end{itemize}
  Then band parallelization is the default.
\end{frame}

\begin{frame}
  \begin{block}{Conclusion}
    Computation of the density from DFT+DMFT is not possible as is because a given
    CPU would need to access arbitrary pairs of bands at a time where bands are
    distributed over CPUs.
  \end{block}
\end{frame}
